\section{A Tutorial Introduction}
	\subsection{Getting started}
		\paragraph{Exersice 1-1}
			Run the "hello, world" program on your system. Experiment with leaving
			out parts of the program, to see what error massages you get.

			\hfill{}\cite[p.~8]{knr}
			\lstinputlisting[title=Exercise 1.1]{../src/ex1_01/main.c}

		\paragraph{Exercise 1-2}
			Experiment to find out what happens when \lstinline{printf}'s argument
			string contains \lstinline{\c}, where \lstinline{c} is some character
			not listed above.

			\hfill{}\cite[p.~8]{knr}
			\lstinputlisting[title=Exercise 1.2]{../src/ex1_02/main.c}

	\newpage
	\subsection{Variables and Arithmetic Expressions}
		\paragraph{Exercise 1-3}
			Modify the temperature conversion program to print a heading above the
			table.

			\hfill{}\cite[p.~13]{knr}
			\lstinputlisting[title=Exercise 1.3]{../src/ex1_03/main.c}

		\paragraph{Exercise 1-4}
			Write a program to print the corresponding Celsius to Farenheit table.

			\hfill{}\cite[p.~13]{knr}
			\lstinputlisting[title=Exercise 1.4]{../src/ex1_04/main.c}

	\newpage
	\subsection{The For Statement}
		\paragraph{Exercise 1-5}
			Modify the temperature conversion program to print the table in reverse
			order, that is, from 300 degrees to 0.

			\hfill{}\cite[p.~14]{knr}
			\lstinputlisting[title=Exercise 1.5]{../src/ex1_05/main.c}

	\newpage
	\subsection{Symbolic Constants}
	There are no exercises in this section.

	\newpage
	\subsection{Character Input and Output}
		\paragraph{Exercise 1-6}
			Verify that the expression \lstinline{getchar() != EOF} is 0 or 1.

			\hfill{}\cite[p.~17]{knr}
			\lstinputlisting[title=Exercise 1.6]{../src/ex1_06/main.c}

		\paragraph{Exercise 1-7}
			Write a program to print the value of EOF.

			\hfill{}\cite[p.~17]{knr}
			\lstinputlisting[title=Exercise 1.7]{../src/ex1_07/main.c}

		\paragraph{Exercise 1-8}
			Write a program to count blanks, tabs and newlines.

			\hfill{}\cite[p.~20]{knr}
			\lstinputlisting[title=Exercise 1.8]{../src/ex1_08/main.c}

		\paragraph{Exercise 1-9}
			Write a program to copy its input to its output, replacing each string
			of one or more blanks by a single blank.

			\hfill{}\cite[p.~20]{knr}
			\lstinputlisting[title=Exercise 1.9]{../src/ex1_09/main.c}

		\paragraph{Exercise 1-10}
			Write a program ro copy its input to its output, replacing each tab by
			\lstinline{\t}, each backspace by \lstinline{\b} and each backslash by
			\lstinline{\\}. This makes tabs and backspaces visible in an unabiguous
			way.

			\hfill{}\cite[p.~20]{knr}
			\lstinputlisting[title=Exercise 1.10]{../src/ex1_10/main.c}

	\newpage
	\subsection{Arrays}
	\subsection{Functions}
	\subsection{Arguments -- Call by Value}
	\subsection{Character Arrays}
	\subsection{External Variables and Scope}
