\section{A Tutorial Introduction}
	\subsection{Getting started}
		\paragraph{Exersice 1-1}
			Run the "hello, world" program on your system. Experiment with leaving
			out parts of the program, to see what error massages you get.

			\hfill{}\cite[p.~8]{knr}
			\lstinputlisting[title=Exercise 1.1]{../src/ex1_01/main.c}

		\paragraph{Exercise 1-2}
			Experiment to find out what happens when \lstinline{printf}'s argument
			string contains \lstinline{\c}, where \lstinline{c} is some character
			not listed above.

			\hfill{}\cite[p.~8]{knr}
			\lstinputlisting[title=Exercise 1.2]{../src/ex1_02/main.c}

	\newpage
	\subsection{Variables and Arithmetic Expressions}
		\paragraph{Exercise 1-3}
			Modify the temperature conversion program to print a heading above the
			table.

			\hfill{}\cite[p.~13]{knr}
			\lstinputlisting[title=Exercise 1.3]{../src/ex1_03/main.c}

		\paragraph{Exercise 1-4}
			Write a program to print the corresponding Celsius to Farenheit table.

			\hfill{}\cite[p.~13]{knr}
			\lstinputlisting[title=Exercise 1.4]{../src/ex1_04/main.c}

	\newpage
	\subsection{The For Statement}
		\paragraph{Exercise 1-5}
			Modify the temperature conversion program to print the table in reverse
			order, that is, from 300 degrees to 0.

			\hfill{}\cite[p.~14]{knr}
			\lstinputlisting[title=Exercise 1.5]{../src/ex1_05/main.c}

	\newpage
	\subsection{Symbolic Constants}
	There are no exercises in this section.

	\newpage
	\subsection{Character Input and Output}
		\paragraph{Exercise 1-6}
			Verify that the expression \lstinline{getchar() != EOF} is 0 or 1.

			\hfill{}\cite[p.~17]{knr}
			\lstinputlisting[title=Exercise 1.6]{../src/ex1_06/main.c}

		\paragraph{Exercise 1-7}
			Write a program to print the value of EOF.

			\hfill{}\cite[p.~17]{knr}
			\lstinputlisting[title=Exercise 1.7]{../src/ex1_07/main.c}

		\paragraph{Exercise 1-8}
			Write a program to count blanks, tabs and newlines.

			\hfill{}\cite[p.~20]{knr}
			\lstinputlisting[title=Exercise 1.8]{../src/ex1_08/main.c}

		\paragraph{Exercise 1-9}
			Write a program to copy its input to its output, replacing each string
			of one or more blanks by a single blank.

			\hfill{}\cite[p.~20]{knr}
			\lstinputlisting[title=Exercise 1.9]{../src/ex1_09/main.c}

		\paragraph{Exercise 1-10}
			Write a program ro copy its input to its output, replacing each tab by
			\lstinline{\t}, each backspace by \lstinline{\b} and each backslash by
			\lstinline{\\}. This makes tabs and backspaces visible in an unabiguous
			way.

			\hfill{}\cite[p.~20]{knr}
			\lstinputlisting[title=Exercise 1.10]{../src/ex1_10/main.c}

		\paragraph{Exersice 1-11}
			How would you test the word count program? What kinds of input are
			most likely	to uncover bugs if there are any?

			\hfill{}\cite[p.~21]{knr}

		\paragraph{Exersice 1-12}
			Write a program that prints its input one word per line.
			
			\hfill{}\cite[p.~21]{knr}

	\newpage
	\subsection{Arrays}
		\paragraph{Exersice 1-13}
			Write a program to print a histogram of the lengths of words in
			its input. It is easy to draw a histogram with the bars horizontal;
			a vertical orientation is more challenging.
	
			\hfill{}\cite[p.~24]{knr}

		\paragraph{Exersice 1-14}
			Write a program to print a histogram of the frequnecies o fdifferent
			characters in its input.
	
			\hfill{}\cite[p.~24]{knr}

	\newpage
	\subsection{Functions}
		\paragraph{Exersice 1-15}
			Rewrite the temperature program conversion program of section 1.2 to
			use a function for conversion.
	
			\hfill{}\cite[p.~27]{knr}

	\newpage
	\subsection{Arguments -- Call by Value}
	There are no exercises in this section.
	
	\newpage
	\subsection{Character Arrays}
		\paragraph{Exersice 1-16}
			Revise the main routine of the longest-line program so it will
			correctly print length of arbitrarily long input lines, and as
			much as possible of the text.
	
			\hfill{}\cite[p.~30]{knr}

		\paragraph{Exersice 1-17}
			Write a program to print all input lines that are longer than 80
			characters.
	
			\hfill{}\cite[p.~31]{knr}

		\paragraph{Exersice 1-18}
			Write a program to remove trailing blanks and tabs from each line
			of input, and to delete entirely blank lines.

			\hfill{}\cite[p.~31]{knr}

		\paragraph{Exersice 1-19}
			Write a function \lstinline{reverse(s)} that reverses the
			character string \lstinline{s}. Use it to write a program that
			reverses its inputs a line at a time.
	
			\hfill{}\cite[p.~31]{knr}

	\newpage
	\subsection{External Variables and Scope}
		\paragraph{Exersice 1-20}
			Write a program \lstinline{detab} that replaces tabs in the input
			with the proper number of blanks to space the next tab stop.
			Assume a fixed set of tab stops, say every $n$ columns. Should $n$
			be a variable or a symbolic parameter?
	
			\hfill{}\cite[p.~34]{knr}

		\paragraph{Exersice 1-21}
			Write a program \lstinline{entab} that replaces strings of blanks
			by the minimum number of tabs and blanks to achieve the same
			spacing. Use the same tab stop as for \lstinline{detab}. When
			either a tab or a single blank would suffice to reach a tab stop,
			which should be given preference?
	
			\hfill{}\cite[p.~34]{knr}

		\paragraph{Exersice 1-22}
			Write a program to ``fold'' long input lines into two or more
			shorter lines after the last non-blank character that occurs
			before the $n$-th column of input. Make sure your program does
			something intelligent with very long lines, and if there are no
			blanks or tabs before the specified column.
	
			\hfill{}\cite[p.~34]{knr}

		\paragraph{Exersice 1-23}
			Write a program to remove all comments from a C program. Don't
			forget to handle quoted strings and character constants properly.
			C comments do net nest.
	
			\hfill{}\cite[p.~34]{knr}

		\paragraph{Exersice 1-24}
			Write a program to check a C program for rudimentary sytax
			errors like unbalanced parentheses, brackets and braces. Don't
			forget about quotes, both single and double, escape sequences,
			and comments. (This program is hard if you do it in full
			generality).
	
			\hfill{}\cite[p.~34]{knr}

