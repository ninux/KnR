\newpage
\section{Make}
	Make is a very powerful and yet easy to use tool for
	processing compilation. Here is a minimalistic tutorial 
	with two simple examples for common usecases like in this
	project.
	\subsection{Variables}
		In a Makefile you can define variables, such as
		for the compiler options. As usual we're going to
		use a appropriate variable name like 
		\lstinline{CFLAGS}. Such a variable can be 
		assigned like this
		\lstinline{CFLAGS = -Wall -Wextra -Wpedantic}.

		To use such a variable you can just set it between
		braces with a prefix dollar symbol like this
		\lstinline{$(FLAGS)}. %$

		This kind of variable can be used for a lot of
		different things like object or library paths
		like \lstinline{OBJECT_PATH = obj/}.

	\subsection{Targets}
		Targets are used to divide a Makefile into different
		sections that can be run separately. A target is
		defined like a label in assemly, that is a string
		at line beginning ended with a colon like 
		\lstinline{all:}. To run a specific section you can 
		give it as parameter to make. For the above exmple
		this would be \lstinline{make all}.

	\subsection{Phony targets}
		A phony target is used for sequences that don't
		produce a file. This is useful for clean-up 
		preocedures. Therefore a common phony target is
		\lstinline{clean}. To define such a phony target
		you can do so with these two steps: First, declare
		a target as phony with \lstinline{.PHONY: clean}.

	\newpage
	\subsection{Examples}
		\subsubsection{Compiling C code with GCC}
			\lstinputlisting[language=make]{../skel/Makefile}
	
		\subsubsection{Building LaTeX projects}
			This example shows how you can use loops in a
			Makefile, which is actually just a bash command.
			This is typically useful for LaTeX documents,
			since it needs up to 3 iterations due to the
			indexing
			\lstinputlisting[language=make]{../tex/Makefile}
